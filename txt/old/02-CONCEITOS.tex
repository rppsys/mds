\section{Conceitos}  \label{sec-conceitos}

\colorbox{red}{ESTE É UM DOCUMENTO RASCUNHO SEM VALIDADE}


\subsection{Atores}

Um ator representa um conjunto coerente de papéis que os usuários do processo desempenham quando de sua execução. Tipicamente, um ator representa um papel que uma entidade desempenha durante a execução
de um processo. Nesse contexto, ele é visto como um conjunto de atribuições, funções e/ou responsabilidades que um ator possui.

\subsubsection{\RSA}

\textbf{Definição}: Unidade administrativa da \CLDF que demanda a solução de tecnologia da informação. 

\subsubsection{\RSI} 

\textbf{Definição}: Servidor representante da Área Requisitante da Solução, indicado pela autoridade competente dessa área, com capacidade técnica relacionada à área de negócio em que a mesma atua. 

\textbf{Pode fazer o papel de Product Owner do SCRUM}

\subsubsection{\TIA}

\textbf{Definição}: Unidade setorial ou seccional do SISP, bem como área correlata, responsável por gerir a Tecnologia da Informação na \CLDF. 

Normalmente, esta área é a Unidade Administrativa Seção de Administração de Sistemas (SEASI) da Coordenadoria de Modernização e Informática (CMI) da CLDF;

\subsubsection{\TII} 

\textbf{Definição}: Servidor representante da Área de Tecnologia da Informação, indicado pela autoridade competente dessa área, com conhecimento técnico relacionado à Solução.

\textbf{Pode fazer o papel de Product Owner do SCRUM}


\subsubsection{\FSA}

\textbf{Definição}: Entidade provedora da Solução de Tecnologia da Informação.

\subsubsection{\FSI}
    \textbf{Definição}: Representante da Fábrica de Software, responsável por atuar como interlocutor principal junto à área de tecnologia de informação, incumbido de receber, diligenciar, encaminhar e responder as principais questões técnicas, legais e administrativas referentes aos softwares que estiverem sendo desenvolvidos.

\subsubsection{\FSDS}
    \textbf{Definição}: Desenvolvedor Sênior da Fábrica de Software que \textbf{normalmente fará o papel de líder (SCRUM MASTER) da equipe de desenvolvimento}. 

\subsubsection{\FSDJ}
    \textbf{Definição}: Desenvolvedor Júnior da Fábrica de Software que integra a \textbf{Equipe de Desenvolvimento do SCRUM} e atua sob a liderança de um Desenvolvedor Sênior.

\subsubsection{\FSDD}
    \textbf{Definição}: Web Designer ou Designer de Telas que integra a \textbf{Equipe de Desenvolvimento do SCRUM} e atua sob a liderança de um Desenvolvedor Sênior.  

\subsubsection{\FQA}

\textbf{Definição}: Entidade responsável por avaliar a qualidade da solução de software.

\subsubsection{\FQI}

\textbf{Definição}: Representante da Fábrica de Qualidade, responsável por atuar como interlocutor principal junto à área de tecnologia de informação, incumbido de receber, diligenciar,
encaminhar e responder as principais questões técnicas, legais e administrativas referentes aos softwares que estiverem sendo avaliados em relação a aspectos de qualidade.

\clearpage
\newpage

% ==========================================================
% ==========================================================
% ==========================================================

\subsection{Artefatos}

Nesta seção apresenta-se os artefatos usados no processo.

% Documento de Oficialização de Demanda de Requisição
\subsubsection{\ADODRF (\ADODRS)}

O \ADODRF deve possuir:
\begin{itemize}
    \item Indicação do \RSI
    \item Depois vou inserindo mais coisas
\end{itemize}


% Documento de Oficialização de Demanda de Software
\subsubsection{\ADODSF (\ADODSS)}

O \ADODSF deve possuir:
\begin{itemize}
    \item Indicação do \TII
    \item Elicitação Inicial
\end{itemize}

Para fazer a Elicitação a \TIA pode usar técnicas de elicitação como por exemplo:

\setlength{\columnsep}{0.2in}
  \begin{multicols}{3}
    \begin{itemize}
        \item Brainstorming
        \item Análise Documental
        \item Grupos Focais
        \item Análise de Interfaces
        \item Entrevistas
        \item Observação
        \item Prototipagem
        \item Workshops de Requisitos
        \item Pesquisa/Questionário
    \end{itemize}
  \end{multicols}