\section{Artefatos}

\subsection{Documento de Requisição}

Este é um documento criado pela Área de Negócios requisitante que demanda um novo sistema ou alteração de um sistea existente.

\begin{env-nota}
Podemos fazer um formulário em WORD com campos que devem ser preenchidos para padronizar esse documento.
\end{env-nota}

O Documento de Requisição deve conter:

\begin{enumerate}
    \item Tipo de Demanda
        \begin{itemize}    
            \item Novo Sistema
                \begin{itemize}    
                    \item Descrição Sucinta do Novo Sistema
                    \item Enquadramento do Novo Sistema Na Classificação ASI 
                    \item Outras coisas ...
                \end{itemize}
            \item Alteração de Sistema Existente
                \begin{itemize}    
                    \item Qual é o Sistema?
                    \item Descrição sucinta da alteração
                    \item Outras coisas ...
                \end{itemize}
        
            
            
        \end{itemize}
    \item Alinhamento ao Plano Setorial
    \item Alinhamento ao PDTI
    \item \textbf{Indicação do Líder de Negócios}
    
\end{enumerate}

\subsection{Justificativa de Aprovação ou Reprovação}
A Área de TI deve analisar o Documento de Requisição com base em critérios de aprovação ou reprovação.

Alguns desses critérios:

\begin{itemize}
    \item Alinhamento ao Plano Setorial
    \item Alinhamento ao PDTI
    \item Viabilidade 
    \item Priorização
    \item Capacidade de Atendimento
    \item Etc...
\end{itemize}

O documento final deve conter:

\begin{itemize}
    \item Conclusão: Aprovado ou Reprovado
    \item Descrição da Justificativa do porque a requisição foi aprovada ou não;
\end{itemize}


