% Seção Introdução


\section{Introdução} \label{sec-intro}

\colorbox{red}{ESTE É UM DOCUMENTO RASCUNHO SEM VALIDADE}

Estou escrevendo o descritivo do processo. Estou me baseando no \citeasnoun{silva2015contrataccao} que é um guia de desenvolvimento de software ágil do SISP, o BABOK, acordões do TCU e exemplo de processos da Camara, Senado, IPHAN e etc...


    \subsection{Papéis SCRUM} 
    
    Estou mapeandos os papéis do SCRUM aos papéis da nossa realidade;

        \subsubsection{Product Owner} Pode ser um financiador ou um importante interessado no projeto, suas principais responsabilidades são:
        \begin{itemize}
            \item Definir as funcionalidades do produto = UA 
            DEMANDANTE + SEASI + FÁBRICA
        	
        	\item Concentra as informações vindas dos usuários = SEASI
        	
        	\item Responsável pelo ROI = SEASI
        	
        	\item Prioriza o Product Backlog = SEASI
        	
        	\item Pode alterar as prioridades dentro do sprint = SEASI
        	
        	\item Aceita ou rejeita os resultados dos trabalhos = SEASI
        \end{itemize}

\colorbox{blue!10}{Então o PO é um servidor da SEASI}

\subsubsection{Scrum Master} Desempenha uma liderança gerenciando os interesses do PO junto ao Time:

\begin{itemize}

	\item Promove a criatividade e o conhecimento no Time. = FÁBRICA DE SOFTWARE
	\item Estimula a comunicação entre todos os envolvidos.  = FÁBRICA DE SOFTWARE
	\item Protege o time de interferências externas.  = FÁBRICA DE SOFTWARE
	\item Remove impedimentos.
	\item Garante que o processo está sendo respeitado.
	\item Gerencias as reuniões (Daily, Sprint Review e Retrospective).
	\item Integra Cliente e Desenvolvimento.
	\item Apoio o PO a maximizar o ROI.
\end{itemize}

\colorbox{blue!10}{Quem é? Desenvolvedor Líder da Fábrica de Software}


\subsubsection{Membros da Equipe} É muito mais um grupo de pessoas do que um papel, são aqueles diretamente ligados ao desenvolvimento do projeto sendo que suas principais características são:

\begin{itemize}
	\item Multi-funcional
	\item Formado por 3 à 9 (Scrum Guide) ou 7 ± 2 pessoas (InfoQ).
	\item Define o objetivo do sprint e especifica os resultados dos trabalhos.
	\item Faz o que é necessário para atingir os resultados.
	\item Auto-Organizável.
	\item Apresenta os resultados do Sprint.
\end{itemize}


\colorbox{blue!10}{Quem são?}

\begin{itemize}
    \item Desenvolvedores da Fábrica de Software
    \item Designer da Fábrica de Software
    \item Testadores da Fábrica de Qualidade
\end{itemize}
