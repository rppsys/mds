\section{Processo} \label{sec-processo}


\colorbox{red}{ESTE É UM DOCUMENTO RASCUNHO SEM VALIDADE}


O processo se divide em Planejamento, Iterações (Sprints) e Implantação. A seguir detalha-se cada fase:


\subsection{Planejamento}


\begin{enumerate}

    \item A \RSA produz o \ADODRF e encaminha para a \TIA. 
    
    \item A \TIA executa: 
    
    \begin{itemize}
        \item Análise de Viabilidade
        \item Priorização
        \item Aprovação ou Arquivamento
    \end{itemize}
    
    Se for viável, o processo segue. Se não, o processo é arquivado. 
    
    \item A \TIA produz o \ADODSF e encaminha para a \FSA

    \begin{itemize}
        \item Neste momento o \TII é indicado.
    \end{itemize}    

\end{enumerate}


    \subsection{Execução}
    
    \colorbox{red!30}{EU AINDA ESTOU TRABALHANDO NESSA PARTE}
    
    \colorbox{red!30}{VOU TRAZER MAIS ELEMENTOS DO SCRUM}    
 
\begin{enumerate}
     
    \item A \FSA em conjunto com a \TIA faz o levantamento dos requisitos, estórias de usuário.
    
    \item A \FSA cria a iteração;

    \item A \FSA fabrica o software;

    \item A \FSA entrega o software para a \TIA aprovar;

    \item A \TIA pode utilizar a \FQA para auxiliar na aprovação ou reprovação do software; Se for aprovado, o software segue para implantação;

\end{enumerate}


    \subsection{Implantação}

\begin{enumerate}
    \item Se o software for aprovado, a \TIA coloca o software em homologação e o processo termina;

    \item Se o software não for aprovado, a \TIA entrega os motivos para a \FSA corrigir os erros. E uma nova sprint é planejada.

\end{enumerate}


