\section{Classificação de Demandas}
\label{sec:cap-demandas}

As demandas por desenvolvimento de sistemas serão divididas em tipos:

\setlength{\columnsep}{0.2in}
\begin{multicols}{2}
    \begin{enumerate}
        \item Sistema Novo
        \item Manutenção Evolutiva
        \item Manutenção Corretiva
        \item Refatoração
    \end{enumerate}
\end{multicols}

\subsection{Sistema novo}

    Serão considerados neste tipo de demanda os projetos de:
    
    \begin{enumerate}
        \item Sistema a ser integralmente desenvolvido;

        \item Sistema reconstruído a partir de um legado;
    
        \item Sistema desenvolvido a partir de outros sistemas pré-existentes, em todo ou em parte, que nunca entraram em produção;
        
        \item Sistema desenvolvido a partir de um ou mais sistemas de outro(s) órgão(s) ou entidade(s), cujo código fonte foi, em todo ou em parte, cedido ou repassado à CLDF, ou obtido por outros meios.
    \end{enumerate}

\subsection{Manutenção evolutiva}

Manutenção evolutiva aborda as mudanças em requisitos funcionais da aplicação, ou seja, a inclusão de
novas funcionalidades, alteração ou exclusão de funcionalidades em aplicações implantadas \cite{roteirosisp2016}. 

A Manutenção Adaptativa, Perfectiva e Cosmética são tipos de Manutenção Evolutiva.

\subsection{Manutenção corretiva}

Manutenção corretiva é a intervenção em um sistema cujas funcionalidades passaram a apresentar defeito,
afetando sua qualidade funcional, executada para mantê-lo em estado operacional. 

É importante destacar
que as demandas por manutenção corretiva precisam ser atendidas com urgência.

Caso o sistema esteja em garantia, sob responsabilidade da empresa que o desenvolveu, esta será
acionada para suas devidas correções, nas condições e prazos estabelecidos; nesse caso, a manutenção
corretiva será considerada acionamento da garantia.

\subsection{Refatoração}

%https://www.trt9.jus.br/pds/pdstrt9/guidances/concepts/refactoring_1B63BA3B.html

A refatoração é uma forma disciplinada de reestruturar o código quando pequenas mudanças são feitas 
nele para melhorar o design. Um aspecto importante de uma refatoração é que ela melhora o design sem mudar o comportamento do design; uma refatoração não adiciona nem remove funcionalidade \cite{fowler2018}. 
Nesta metodologia, a refatoração é uma demanda de adequação do sistema, 
a fim de alterar uma funcionalidade que já foi implementada, entregue e validada. 
A alteração ou adequação do sistema para finalizar uma funcionalidade que necessite ser 
implementada 
em mais de uma Sprint não será considerada uma refatoração.
    