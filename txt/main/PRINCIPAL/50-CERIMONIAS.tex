\section{Reuniões} 
\label{sec:cap-reunioes}

    \begin{wrapfigure}{r}{0.25\textwidth}
    \centering
    \includegraphics[width=0.25\textwidth]{fig/cer-reunioes.jpg}
    \end{wrapfigure}

    O PDAS apresenta diversas reuniões. Duas delas são reuniões que ocorrem nas fases de Planejamento e Encerramento do macro processo principal. As \Sprints ocorrem na fase de desenvolvimento e, em cada \Sprint deverão ocorrer duas reuniões: uma para planejar a \Sprint e outra para encerrar. 
    
    Essa seção visa descrever cada uma dessas reuniões.

% ---------------------------------------------
    \subsection{Reuniões de Planejamento}
    \label{sec:reuniaoPlanejamento}
% ---------------------------------------------

    As reuniões de planejamento são talvez as reuniões mais importantes do processo pois são nelas que as regras do jogo serão definidas e compreendidas por todos os atores antes de se iniciar o desenvolvimento do software propriamente dito.  
    
    Nessas reuniões, uma série de artefatos mínimos deverão ser elaborados para se seguir para a fase de desenvolvimento. 
    
    As reuniões de planejamento são apresentadas em um processo próprio chamado ``Realizar Reuniões de Planejamento'', que por sua vez é dividido em dois subprocessos:
    
    \begin{enumerate}
        \item Processo ``Realizar Reuniões entre TI e Negócios''
    
        \item Processo ``Realizar Reuniões entre CLDF e Fábricas''
    \end{enumerate}
    
    O leitor já deve ter compreendido que existem dois tipos de reuniões que devem acontecer: o primeiro tipo envolve apenas integrantes das áreas internas da CLDF. E o segundo tipo vai envolver essas áreas e as fábricas. 
    
    \subsubsection{Realizar Reuniões entre TI e Negócios}
    
    As reuniões entre as áreas de TI e Negócios existem para que os seguintes artefatos sejam elaborados, de preferência nesta ordem:
    
    \begin{enumerate}
        \item \textbf{\nameref{sec:art-docdefltln}} - O primeiro artefato que deve ser elaborado é o \nameref{sec:art-docdefltln} descrito na seção \ref{sec:art-docdefltln}, ou seja, as áreas de TI e de negócios devem realizar tantas reuniões quanto 
        forem necessárias para definir esses atores. 
    
        \item \textbf{\nameref{sec:art-docvisao}} - Em seguida esses líderes devem se reunir para elaborar o ``Documento de Visão'', artefato descrito na seção \ref{sec:art-docvisao}.
        
        \item \textbf{\nameref{sec:art-pb} inicial} - Finalmente, os líderes já podem começar a desenvolver um \PB inicial de acordo com o descrito na seção \ref{sec:art-pb} criando e adicionando ``Histórias de Usuário'' (seção \ref{sec:art-us}) a ele. Nesse momento, é importante já configurar o ambiente de gerenciamento das \Sprints que será utilizado (Redmine, etc..). 
    \end{enumerate}
    
    Assim, quando for o momento de envolver as fábricas, os líderes já terão uma compreensão mais consistente do projeto que irão desenvolver.
    
    \subsubsection{Realizar Reuniões entre CLDF e Fábricas}
    
    Após as reuniões internas e a criação dos artefatos enumerados na seção anterior, os líderes já possuem uma visão mais consistente do projeto que pretendem desenvolver. Dessa forma, chega a hora de realizar reuniões com as fábricas. Assim, os líderes devem convocar os representantes das fábricas para reuniões com os seguintes objetivos:
    
        \subsubsubsection{Objetivo: Apresentar Líderes CLDF e Representantes Fábricas}
    
                Inicialmente os atores devem se apresentar e se conhecer. Dessa forma o objetivo é apresentar quem serão os Líderes Técnico e de Negócios da CLDF e quem são os Representantes das Fábricas de Software e Qualidade

        \subsubsubsection{Objetivo: Apresentar este MDS e o PDAS}

            
                Uma cópia deste documento já pode ter sido enviada para as Fábricas tomarem conhecimento do processo esperado antes mesmo da reunião.


        \subsubsubsection{Objetivo: Apresentar o ``Documento de Visão'' do projeto que será desenvolvido}

                O ``Documento de Visão'' deve ser apresentado para os representantes das Fábricas de Software e Qualidade.  
            

        \subsubsubsection{Objetivo: Elaborar e aprovar o ``Documento de Definição de Artefatos''}

            
                Naturalmente, um projeto mais complexo terá artefatos que um projeto mais simples não precisa ter. Dessa forma, diferentes projetos podem ter diferentes conjunto de artefatos listados no \textbf{Documento de Definição de Artefatos} elaborado nessas reuniões.
    
                Os líderes e as fábricas precisam definir os artefatos que serão necessários em cada fase do projeto (planejamento, desenvolvimento e encerramento). As Fábricas de Software e de Qualidade poderão propor artefatos diversos, mas cabe aos Líderes Técnico e de Negócios definir quais serão exigidos ou não e quem são os responsáveis pela elaboração desses artefatos.     
                
                A lista de artefatos deve ser formalizadas no ``Documento de Definição de Artefatos'' que explicita que artefatos serão exigidos em cada fase do processo e quem são os responsáveis por sua elaboração. A seção ``\nameref{sec:art-sug}'' (seção \ref{sec:art-sug}) apresenta uma série de artefatos que podem ser utilizados nos projetos.
            
        \subsubsubsection{Objetivo: Elaborar e aprovar a ``Forma da Entrega'' dos entregáveis}
    
                A ``\nameref{sec:art-fe}'' é apresentada na seção \ref{sec:art-fe}. Trata-se de um acordo entre CLDF e as Fábricas da forma que cada entregável deve ser entregue pelos responsáveis por sua elaboração de forma que os líderes possam fazer o ateste técnico.
                
                Recomenda-se a forma de entrega de cada artefato seja formalizada no mesmo  ``Documento de Definição de Artefatos'' elaborado anteriormente.   
            

        \subsubsubsection{Objetivo: Elaborar e aprovar a ``Definição de Pronto''}

                A \nameref{sec:art-defpronto} é apresentada com detalhes na seção \ref{sec:art-defpronto}. Trata-se de um documento que deve ser elaborado em conjunto com as fábricas de forma que todos tenham o mesmo entendimento do que significa um trabalho realizado durante a fase de desenvolvimento (as \Sprints) estar ``Pronto''.
            

        \subsubsubsection{Objetivo: Determinar a ``Duração da \Sprint''}

                Finalmente, a duração da \Sprint deve ser acordada entre os envolvidos. Recomenda-se que seja escolhida a duração de 2 ou 3 semanas. Projetos mais complexos podem necessitar de \Sprints de 3 semanas enquanto projetos mais simples podem ser realizados em \Sprints de menor duração.
            
        \subsubsubsection{Objetivo: Amadurecer o \PB}
        
                O \PB inicial elaborado nas reuniões internas anteriores (entre as Áreas de TI e Negócios) deve ser apresentado para a Fábrica de Software e esta deve colaborar com seu desenvolvimento propondo novos itens e ``Histórias de Usuário''. Esses itens serão avaliados e priorizados pelo Líder de Negócios. Dessa feita, quando o processo seguir para a fase de desenvolvimento, o \PB já apresentará ``Histórias de Usuário'' consistentes priorizadas que poderão ser desenvolvidas.

% ---------------------------------------------
    \subsection{Reuniões de Desenvolvimento}
    \label{sec:reuniaoDesenvolvimento}
% ---------------------------------------------

    As Reuniões de Desenvolvimento são as que ocorrem dentro das \Sprints. Uma vez que uma \Sprint começa, sua duração é fixada e não pode ser reduzida ou aumentada. Os eventos restantes podem terminar sempre que o propósito do evento é alcançado, garantindo que uma quantidade adequada de tempo seja gasta sem permitir perdas no processo. 
    
    \subsubsection{Reunião de Planejamento da \Sprint}
    \label{sec:reuniaoPlanejamentoSprint}
    
    Sempre antes de cada nova \Sprint deve ocorrer a Reunião de Planejamento da \Sprint onde o Time \SCRUM deve realizar o planejamento do que deverá ser entregue ao final do ciclo da \Sprint que está prestes a ser realizada. 
    
    Essa reunião é divida em 2 fases: Fase Inicial e Fase Final cada uma com no máximo 4 horas de duração.
    
    \subsubsubsection{Fase Inicial}
    
    A fase inicial deve responder à pergunta: \textbf{O que pode ser ``Pronto'' nesta Sprint?} 
    
    Para responder a essa pergunta, sugere-se a seguinte ordem de ações na reunião a serem desempenhadas por cada ator:
    
    \begin{enumerate}
        \item O Líder de Negócios (\PO) auxiliado pelo Líder Técnico deve definir a Meta da \Sprint e expor ao Time de Desenvolvimento os itens mais prioritários do \PB.
        
        \item O Time de Desenvolvimento deve estimar os itens em tamanho e definir o que acredita que pode ser implementado dentro da \Sprint. Essa selação é chamada de \emph{Selected Product Backlog}.
    \end{enumerate}
    
    \subsubsubsection{Fase Final}
    
    A fase final deve responder à pergunta: \textbf{Como o trabalho necessário para entregar o incremento será realizado?}
    
    Da mesma forma, sugere-se a seguinte ordem de ações:
    
    \begin{enumerate}
        \item O Time de Desenvolvimento colhe mais detalhes do 
        \emph{Selected Product Backlog} decompondo-os em tarefas gerando assim o \Sprint \emph{backlog}.
        
        \item Após isso, cada membro do time de desenvolvimento deve selecionar as atividades que deseja executar na \Sprint e estimá-la em horas.
    \end{enumerate}
    
    \subsubsection{Reunião Diária}
    
    É um evento de 15 minutos, para que o Time de Desenvolvimento possa sincronizar as atividades e criar um plano para as próximas 24 horas. É uma reunião onde cada participante fala sobre o progresso conseguido, o trabalho a ser realizado e/ou o que o impede de seguir avançando. É mantida no mesmo horário e local todos os dias para reduzir a complexidade. O Scrum Master assegura que o Time de Desenvolvimento tenha a reunião, mas o Time de Desenvolvimento é responsável por conduzir a Reunião Diária. 
    
    \subsubsection{Reunião de Encerramento da Sprint}
    \label{sec:reuniaoEncerramentoSprint}

    Nesta reunião, com duração máxima de 4 horas, deve-se realizar uma reunião com o propósito de verificar os resultados da \Sprint e fazer uma retrospectiva dos aspectos que podem ser melhorados na próxima \Sprint.
    
    De forma resumida, sugere-se que no mínimo as seguintes ações sejam realizadas:
    
    \begin{enumerate}
        \item O Time de Desenvolvimento deve apresentar todos os resultados para o \PO;
        
        \item O \PO avalia se a Meta da Sprint foi ou não atingida.
        
    \end{enumerate}

    No entanto, essa reunião também destina-se a motivar, obter feedback e promover a colaboração entre todas as partes envolvidas. Dessa forma, a Reunião de Encerramento da Sprint também pode conter os seguintes elementos:


    \begin{itemize}
    
        \item Além do Time Scrum, os \emph{Stakeholders} chave convidados pelo \PO podem ser incluídos para participar dessa reunião;

        \item  O Time de Desenvolvimento discute o que foi bem durante a Sprint, quais problemas ocorreram dentro da Sprint, e como estes problemas foram resolvidos;

        \item   O grupo todo colabora sobre o que fazer a seguir, e é assim que a Revisão da Sprint fornece valiosas entradas para o Planejamento da Sprint subsequente;

        \item  O \PO pode fazer anotações que podem se tornar novos itens no \PB.
    
    \end{itemize}
    
    
% ---------------------------------------------
    \subsection{Reunião de Encerramento}
    \label{sec:reuniaoEncerramento}
% ---------------------------------------------
    
    A reunião de encerramento deve ocorrer no momento em que for decidido que não haverão mais \Sprints. Nessa reunião as Fábricas deverão entregar quaisquer artefatos de encerramento que tiverem sido definidos nas reuniões de planejamento realizadas no início do projeto. Geralmente, uma única reunião será necessária para isso, mas caso no ateste técnico se verifique que ainda há pendências, outras reuniões podem ser convocadas para realizar o encerramento formal.
    
    A reunião de encerramento também é o momento em que os Líderes podem solicitar atividades de encerramento como treinamentos e etc.
        