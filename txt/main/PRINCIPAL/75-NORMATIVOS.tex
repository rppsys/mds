\clearpage
\newpage

\section{Normas Complementares}
\label{sec:cap-normas}

\subsection{Normas Complementares para Contagem de Pontos de Função} 

As disposições a seguir detalham a execução prática do processo
\hyperlink{tar:det-rat-pro-esms}{Executar Serviço de Medição de Sistemas} (seção \ref{tar:det-rat-pro-esms}) sem alteração de seu fluxo formal no PDAS.


\subsubsection{Divergência de Aferição de Pontos de Função}

Se houver divergência em relação a contagem dos Pontos de Função apresentada pela Fábrica de Software (FS), esta deverá apresentar a contagem nos pontos de divergência, de forma detalhada, com base nos critérios de identificação de Processo Elementar definidos no CPM 4.3.1, e o papel do usuário envolvido ``por padrão'', quando houver um fluxo de trabalho de múltiplos papéis, deverá ser destacado.

% Texto  escrito pela Ana Clélia em 23/02/2026 pelo Teams


\subsubsection{Reunião de Mediação de Divergência de Aferição de Pontos de Função}

Nos casos em que permanecer a divergência quanto à aferição de contagem de pontos de função da Fábrica de Software, reunião técnica de mediação poderá ser convocada.

\textbf{Participantes obrigatórios:}
\begin{itemize}
	\item Especialista em Métricas da Fábrica de Software;
	\item Especialista em Métricas da Fábrica de Métricas;
	\item Especialista em Negócio da CLDF.
\end{itemize}

\textbf{Participação excepcional:}

A participação de outros profissionais das partes somente poderá ocorrer:
\begin{itemize}
	\item mediante justificativa técnica formal;
	\item em caráter excepcional;
	\item com autorização prévia da CLDF;
	\item desde que demonstrada relevância técnica para análise da contagem ou conhecimento do negócio envolvido.
\end{itemize}

\subsubsection{Decisão}

Caso a divergência permaneça quanto a aferição da contagem de PF detalhada apresentada pela Fábrica de Software, a decisão final caberá a CONTRATANTE.







