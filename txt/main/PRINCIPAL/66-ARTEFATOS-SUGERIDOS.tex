\subsection{Artefatos sugeridos}
\label{sec:art-sug}

    Conforme foi explicado, diferentes projetos podem exigir diferentes conjuntos de artefatos. Essa seção apresenta uma lista de artefatos sugeridos categorizados de acordo com as fases em que eles podem ser usados. Também sugerimos quem seriam os responsáveis pela elaboração desses artefatos.
    
% ##################################    
\subsubsection{Fase de Planejamento}
% ##################################

A tabela \ref{tab:art-plan} presenta um conjunto de artefatos que poderiam ser exigidos na fase de planejamento do PDAS.

\begin{table}[!h]
    \begin{center}
    \begin{tabular}{|p{0.4\textwidth}|*{5}{p{0.1cm}}|}
        \hline
            \rowcolor{blue!10} \multicolumn{6}{|c|}{\large Artefatos x Responsável (Fase de Planejamento) \normalsize} \\ \hline \hline
            
            % CABEÇALHO        
            \rowcolor{lightgray!30}
            \textbf{ARTEFATOS}
            & \textbf{LN} & \textbf{LT} & \textbf{FS} & \textbf{FQ} & \textbf{FM} \\ \hline
            % -----            

            % CONTEÚDO        
            \rowcolor{\corLN}
            Glossário de termos técnicos do negócio
            & \msmark & \msnone & \msnone & \msnone & \msnone \\ \hline
            % -----            
            \rowcolor{\corFS}
            Proposta de Identidade Visual
            & \msnone & \msnone & \msmark & \msnone & \msnone \\ \hline
            % -----            
            \rowcolor{\corFQ}
            Estratégia de Teste
            & \msnone & \msnone & \msnone & \msmark & \msnone \\ \hline
            % -----            
            \rowcolor{\corFQ}
            Relatório de Verificação de Requisitos
            & \msnone & \msnone & \msnone & \msmark & \msnone \\ \hline
            % -----            
    \end{tabular}    
    \caption{\label{tab:art-plan} Artefatos sugeridos para a fase de planejamento}
    \end{center}
\end{table}


% ##################################    
\subsubsection{Fase de Desenvolvimento}
% ##################################    

Na fase de desenvolvimento do PDAS, além do ``Incremento do Produto'', outros artefatos poderiam ser exigidos das Fábricas de Software e Qualidade durante a execução de suas atividades. Uma lista de artefatos sugeridos são apresentados na tabela \ref{tab:art-des}.


\begin{table}[!h]
    \begin{center}
    \begin{tabular}{|p{0.5\textwidth}|*{5}{p{0.1cm}}|}
        \hline
            \rowcolor{blue!10} \multicolumn{6}{|c|}{\large Artefatos x Responsável (Fase de Desenvolvimento) \normalsize} \\ \hline \hline
            
            % CABEÇALHO        
            \rowcolor{lightgray!30}
            \textbf{ARTEFATOS}
            & \textbf{LN} & \textbf{LT} & \textbf{FS} & \textbf{FQ} & \textbf{FM} \\ \hline
            % -----            
            % CONTEÚDO        
            % -----       
            \rowcolor{\corFS}
            Gráficos de Acompanhamento do Trabalho
            & \msnone & \msnone & \msmark & \msnone & \msnone \\ \hline
            % -----            
            \rowcolor{\corFS}
            Relatório Gerencial
            & \msnone & \msnone & \msmark & \msnone & \msnone \\ \hline
            % -----            
            \rowcolor{\corFS}
            Documentos de Regras de Negócio
            & \msnone & \msnone & \msmark & \msnone & \msnone \\ \hline
            % -----            
            \rowcolor{\corFS}
            Código fonte 
            & \msnone & \msnone & \msmark & \msnone & \msnone \\ \hline
            % -----            
            \rowcolor{\corFS}
            Código compilado e/ou executável
            & \msnone & \msnone & \msmark & \msnone & \msnone \\ \hline
            % -----            
            \rowcolor{\corFS}
            Pacote com testes unitários e de integração
            & \msnone & \msnone & \msmark & \msnone & \msnone \\ \hline
            % -----            
            \rowcolor{\corFS}
            Evidências de Testes
            & \msnone & \msnone & \msmark & \msnone & \msnone \\ \hline
            % -----            
            \rowcolor{\corFS}
            Modelo de Entidade de Relacionamento
            & \msnone & \msnone & \msmark & \msnone & \msnone \\ \hline
            % -----            
            \rowcolor{\corFS}
            Planilha de Contagem Detalhada
            & \msnone & \msnone & \msmark & \msnone & \msnone \\ \hline
            % -----            
            \rowcolor{\corFS}
            Manual do sistema
            & \msnone & \msnone & \msmark & \msnone & \msnone \\ \hline
            % -----            
            \rowcolor{\corFS}
            Manual do usuário
            & \msnone & \msnone & \msmark & \msnone & \msnone \\ \hline
            % -----            
            \rowcolor{\corFS}
            Ajuda do sistema
            & \msnone & \msnone & \msmark & \msnone & \msnone \\ \hline
            % -----            
            \rowcolor{\corFQ}
            Cenários e Casos de Teste
            & \msnone & \msnone & \msnone & \msmark & \msnone \\ \hline
            % -----            
            \rowcolor{\corFQ}
            Roteiro de Teste
            & \msnone & \msnone & \msnone & \msmark & \msnone \\ \hline
            % -----            
            \rowcolor{\corFQ}
            Relatório de Não Conformidade (Evidência de Defeito)
            & \msnone & \msnone & \msnone & \msmark & \msnone \\ \hline
            % -----            
            \rowcolor{\corFQ}
            Evidência de Teste
            & \msnone & \msnone & \msnone & \msmark & \msnone \\ \hline
            % -----            
    \end{tabular}    
    \caption{\label{tab:art-des} Artefatos sugeridos para a fase de desenvolvimento}
    \end{center}
\end{table}

    Dentre os diversos artefatos sugeridos, alguns merecem destaque e são melhor apresentados na seções a seguir:

% -----------------------------------------
    \subsubsubsection{Gráficos de Acompanhamento do Trabalho}
    \label{sec:art-grafacompanhatrab}
% -----------------------------------------

    Os Gráficos de Acompanhamento do Trabalho são ferramentas para a visualização do progresso do cumprimento de uma quantidade mensurável estimada de trabalho em um determinado tempo. Eles servem para criar visibilidade de uma forma simples, ajudando a rapidamente identificar problemas e, assim, reduzir os riscos do projeto.
    
    Os Gráficos de Release Burndown e de Sprint Burndown são os dois tipos de Gráficos a serem utilizados. 

    Esses gráficos têm como objetivo mostrar o esforço restante para a conclusão da Sprint/Release, bem como mostrar o quão próximo ou distante a equipe está de atingir a meta da Sprint ou do release.

    Neste gráfico, tem-se uma coluna vertical, que representa a quantidade de esforço (quantidade de trabalho que ainda precisa ser realizada) e uma coluna horizontal, representando as unidades de tempo, como, por exemplo, as horas ou dias para finalizar uma Sprint.
    
    Conforme a figura abaixo, uma linha na cor verde indica o fluxo ideal de trabalho. Quando o gráfico está acima da linha verde, a equipe de desenvolvimento está distante de atingir a meta. Em cima da linha verde, a equipe de desenvolvimento está em um fluxo de trabalho ideal.
    
    
    \begin{figure}[htpb]
    \begin{center}
        \includegraphics[scale=0.5]{fig/art-grafburndown1.jpg}
    \end{center}
    \caption{Gráfico Burndown da \Sprint}
    \label{fig:grafburndown1}
    \end{figure}   
    
    
    
    \begin{figure}[htpb]
    \begin{center}
        \includegraphics[scale=0.5]{fig/art-grafburndown2.jpg}
    \end{center}
    \caption{Gráfico Burndown da \emph{Release}}
    \label{fig:grafburndown2}
    \end{figure}   
    
    
% -----------------------------------------
   \subsubsubsection{Relatório Gerencial}
   \label{sec:art-relatoriogerencial}
% -----------------------------------------   
   
   Documento que contém a lista de artefatos elaborados na Release e é criado na fase de Preparação e atualizado a cada encerramento de Sprint e/ou de Release. 
   
   Recomenda-se que um relatório gerencial inclua os seguintes artefatos: 
   
    \setlength{\columnsep}{0.2in}
    \begin{multicols}{2}
        \begin{itemize}
            \item Backlog do Produto priorizado e segregado em Sprints;
            \item Sprint Backlog e Definição de Pronto (DoD);
            \item Histórias de Usuários detalhadas;
            \item Regras de Negócio;
            \item Protótipo;
            \item Atas de Reunião;
            \item Documento de Arquitetura;
            \item Modelo de Dados;
            \item Plano de Implantação;
            \item Relatório de Testes;
            \item Relatório de Qualidade de Código;
            \item Relatório de Segurança do sistema;
            \item Incremento de Produto (se for o caso);
            \item Manual do Usuário (se for o caso);
            \item Documento de Lições Aprendidas na Sprint e na Release;
        \end{itemize}
    \end{multicols}




% ##################################    
\subsubsection{Fase de Encerramento}
% ##################################    

A reunião de encerramento do projeto é o momento ideal para exigir alguns outros artefatos das fábricas. Exemplo de artefatos que poderiam ser exigidos são apresentados na tabela \ref{tab:art-enc}.

\begin{table}[!h]
    \begin{center}
    \begin{tabular}{|p{0.4\textwidth}|*{5}{p{0.1cm}}|}
        \hline
            \rowcolor{blue!10} \multicolumn{6}{|c|}{\large Artefatos x Responsável (Fase de Encerramento) \normalsize} \\ \hline \hline
            
            % CABEÇALHO        
            \rowcolor{lightgray!30}
            \textbf{ARTEFATOS}
            & \textbf{LN} & \textbf{LT} & \textbf{FS} & \textbf{FQ} & \textbf{FM} \\ \hline
            % -----            
            % CONTEÚDO        
            % -----            
            \rowcolor{\corFS}
            Plano de Treinamento
            & \msnone & \msnone & \msmark & \msnone & \msnone \\ \hline
            % -----     
            \rowcolor{\corFS}
            Material Didático de Treinamento 
            & \msnone & \msnone & \msmark & \msnone & \msnone \\ \hline
            % -----     
            \rowcolor{\corFS}
            Relatório de Lições Aprendidas
            & \msnone & \msnone & \msmark & \msnone & \msnone \\ \hline
            % -----     
            \rowcolor{\corFQ}
            Sumário de Testes
            & \msnone & \msnone & \msnone & \msmark & \msnone \\ \hline
            % -----     
    \end{tabular}    
    \caption{\label{tab:art-enc} Artefatos sugeridos para a fase de encerramento}
    \end{center}
\end{table}

\subsubsection{Artefatos de Medição de Sistemas}

O Processo ``Executar Serviço de Medição de Sistema'' deverá ocorrer em diversos momentos durante a fase de desenvolvimento do software. Assim, alguns artefatos de medição poderão ser exigidos da Fábrica de Métricas. Esses artefatos são apresentados na tabela \ref{tab:art-med}.

\begin{table}[!h]
    \begin{center}
    \begin{tabular}{|p{0.4\textwidth}|*{5}{p{0.1cm}}|}
        \hline
            \rowcolor{blue!10} \multicolumn{6}{|c|}{\large Artefatos x Responsável (Medição de Sistemas) \normalsize} \\ \hline \hline
            
            % CABEÇALHO        
            \rowcolor{lightgray!30}
            \textbf{ARTEFATOS}
            & \textbf{LN} & \textbf{LT} & \textbf{FS} & \textbf{FQ} & \textbf{FM} \\ \hline
            % -----            
            % CONTEÚDO        
            % -----            
            \rowcolor{\corFM}
            Planilha de Contagem
            & \msnone & \msnone & \msnone & \msnone & \msmark \\ \hline
            % -----     
            \rowcolor{\corFM}
            Sumário de Contagem
            & \msnone & \msnone & \msnone & \msnone & \msmark \\ \hline
            % -----     
    \end{tabular}    
    \caption{\label{tab:art-med} Artefatos sugeridos para acompanhar os processos de medição dos sistemas}
    \end{center}
\end{table}


    \subsubsection{Artefatos adicionais}
    
    Alguns artefatos podem ser utilizados em todas as fases do processo. É o caso do ``Termo de Aceite de Fase'' e das atas de reunião. 

    \subsubsubsection{Termo de Aceite da Fase (TAF)}
    
    O termo de aceite da fase é um documento que formaliza o aceite de uma fase. Trata-se de um marco que encerra a fase atual e inicia a fase seguinte.
    
    
    \subsubsubsection{Atas de Reunião}
    
    Uma ata de reunião, ou simplesmente ata, é um registro dos eventos importantes que ocorreram em uma reunião deliberativa. São tradicionalmente feitos por meio de escrita e durante o andamento da própria reunião, simultaneamente à ocorrência desses eventos. Recomenda-se que as deliberações realizadas nas reuniões sejam formalizadas por meio de atas de reunião.  
    
    