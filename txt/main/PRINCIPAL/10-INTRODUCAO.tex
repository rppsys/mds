\clearpage
\newpage

\section{Introdução}
\label{sec:cap-intro}

Uma metodologia pode ser definida como um conjunto de regras e procedimentos estabelecidos para se atingir um determinado fim. Do mesmo modo, uma Metodologia de Desenvolvimento de Sistemas (MDS) é uma coleção de conceitos e boas práticas em desenvolvimento de sistemas. O objetivo é apresentar os conceitos e estabelecer o processo necessário para guiar o desenvolvimento e a manutenção de sistemas corporativos. 

Esse documento visa estabelecer o processo de interação entre a Câmara Legislativa do Distrito Federal e as Fábricas de Software, Qualidade e Métricas contratadas para fazer o desenvolvimento e manutenção de sistemas, avaliação de aspectos de qualidade e a medição dos produtos desenvolvidos, respectivamente, com o objetivo de   
entregar soluções corretas, com qualidade e no prazo que atendam aos requisitos de negócio no âmbito dessa instituição. 
Portanto, além dos conceitos e definições, o principal produto da metodologia é o Processo de Desenvolvimento Ágil de Software (PDAS). Trata-se de um fluxo de processos e subprocessos que deverão guiar essa interação. 

Esse documento foi escrito com base em outras metodologias elaboradas por orgãos e autarquias do governo que também optaram pela estratégia de \emph{outsourcing}. Em particular, o processo foi adaptado da Metodologia Midas \cite{iphan2013} juntamente com elementos de metodologias de desenvolvimento ágil do Senado Federal, da Câmara dos Deputados e da Agência Nacional de Energia Elétrica (ANEEL). Também tomamos o cuidado de pesquisar a jurisprudência de orgãos de controle identificando acórdãos e atas sobre o tema de forma a não repetir práticas inadequadas.  Além disso, o desenvolvimento levou em consideração os elementos de mitigação de riscos da análise realizada na fase de planejamento da contratação. O resultado é uma metodologia desenvolvida à luz do modelo ágil \Scrum, mas adaptada e adequada à realidade institucional da Câmara Legislativa do Distrito Federal. \nocite{atatcu2013,tcu2015}

Assim, iniciamos o documento nesse capítulo \ref{sec:cap-intro} onde introduzimos o que é uma Metodologia de Desenvolvimento de Sistemas (MDS) e qual é o seu propósito. O capítulo \ref{sec:cap-conceitos} apresenta os principais conceitos envolvidos: descrevemos o que são modelos ágeis e explicamos como funciona o \emph{framework} original \Scrum.
Em seguida, o Fluxo do Processo é apresentado no capítulo \ref{sec:cap-fluxo}. Neste momento, apresentamos o Processo de Desenvolvimento Ágil de Software (PDAS) e também os subprocessos que o integram. Os demais capítulos apresentam os elementos necessários para entender o processo como um todo. Dessa forma, no capítulo \ref{sec:cap-demandas}, as demandas por desenvolvimento de sistemas são classificadas. 

Desse ponto segue-se identificando os principais atores do processo, as reuniões e os principais artefatos utilizados (capítulos \ref{sec:cap-atores}, \ref{sec:cap-reunioes} e \ref{sec:cap-artefatos}). Em seguida, o capítulo \ref{sec:cap-detalhes} consolida o entendimento detalhando cada atividade que compõe o processo.


Finalmente, o capítulo \ref{sec:cap-normas} dispõe sobre as Normas Complementares ao PDAS, detalhando procedimentos operacionais e regras de execução contratual aplicáveis aos processos e subprocessos da metodologia. O capítulo \ref{sec:cap-conclusoes} apresenta as considerações finais e, ao final, o anexo contendo os Documentos de Diretrizes consolida os padrões técnicos e normativos aplicáveis ao desenvolvimento, implantação e manutenção de sistemas no âmbito institucional.










