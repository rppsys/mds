\subsection{Artefatos mínimos do processo}
\label{sec:art-min}

    % -----------------------------------------------------    
    \subsubsection{Documento de Definição de Líder Técnico e Líder de Negócios}
    \label{sec:art-docdefltln}
    % -----------------------------------------------------    
    Documento que formaliza, para determinado projeto, quem da Área de TI da CLDF fará o papel de Líder Técnico e quem da Área de Negócios da CLDF fará o papel de Líder de Negócios. Esse documento deve ser elaborado nas reuniões de planejamento.

    % -----------------------------------------------------    
    \subsubsection{Documento de Definição de Artefatos}
    \label{sec:art-docdefartefatos}
    % -----------------------------------------------------    
    Documento que lista quais artefatos serão exigidos em cada fase do PDAS (planejamento, desenvolvimento e encerramento) e quem é o responsável por sua elaboração. Esse documento é criado nas reuniões de planejamento. A seção \ref{sec:art-sug} apresenta uma série de artefatos que podem ser usados nos projetos em cada fase e quais seriam os responsáveis por sua elaboração.
    
% -----------------------------------------
    \subsubsection{Forma das Entregas}
    \label{sec:art-fe}
% -----------------------------------------
    
    Uma entrega é um bem ou serviço tangível ou intangível produzido como resultado de um projeto que se destina a ser entregue a um cliente. Uma entrega pode ser um relatório, um documento, um produto de software, uma atualização do servidor, um pacote de produtos, ou qualquer outro componente de um projeto geral. 
    
    Durante a realização das \Sprints o entregável mais comum é o ``Incremento do Produto'', mas podem ocorrer casos em que sejam necessários entregar outros artefatos, definidos previamente, nas reuniões de planejamento.
    
    Assim, a forma da entrega de um entregável particular deve ser definida. A pergunta que deve ser respondida é:
    
    \begin{quote}
        ``\emph{Como o entregável será entregue?}''
    \end{quote}
    
    Por exemplo, os Líderes podem definir que a forma de entregar o ``Incremento de Produto'' seja por meio da instalação do software em um ambiente de homologação disparando-se um e-mail para o Líder Técnico assim que essa instalação tenha sido realizada. Dessa forma, é importante definir, para cada entregável, a forma da entrega.

% -----------------------------------------
    \subsubsection{Pacote de Produtos de Testes e Controle de Qualidade}
    \label{sec:art-ppcq}
% -----------------------------------------

    Um pacote de produtos de testes de sistemas e controle de qualidade é um conjunto de entregáveis destinados à apuração e comprovação de que um determinado entregável atende ou não a determinados critérios de qualidade.

% -----------------------------------------
    \subsubsection{Pacote de Produtos de Medição de Sistemas}
    \label{sec:art-ppms}
% -----------------------------------------

    Um pacote de produtos de medição de sistemas é um conjunto de entregáveis destinados a medir a quantidade de pontos de função de um determinado produto segundo uma métrica e um sistema de medição previamente definido.  Além da medição, outros relatórios auxiliares podem fazer desse pacote.