\section{Fluxo do Processo}
\label{sec:cap-fluxo}

As próximas seções irão apresentar o fluxo do Processo de Desenvolvimento Ágil de Software para guiar a interação entre os atores da CLDF e das Fábricas contratadas. Iniciamos apresentando o macro fluxo principal e depois apresentamos os subprocessos envolvidos.

\begin{landscape}
    \begin{figure}[htbp]
    \thispagestyle{empty}
    \begin{center}
        \includegraphics[scale=0.58]{fig/BPMN-PDAS.png}
    \end{center}
    \caption{Processo de Desenvolvimento Ágil de Software (PDAS)}
    \label{fig-pgds}
    \end{figure}
\end{landscape}

\subsection{Processo de Desenvolvimento Ágil de Software}
\label{sec:fluxo-pgds}

A figura \ref{fig-pgds} apresenta o \PDAS que se divide nas fases de Planejamento, Desenvolvimento e Encerramento.

O processo se inicia na Fase de Planejamento com o ``Recebimento da Demanda'' e segue para o ``\nameref{sec:fluxo-rrp}'' onde as reuniões de planejamento são realizadas e os artefatos de planejamento são elaborados e definidos.

Depois ele segue para o ``\nameref{sec:fluxo-rat}'' onde os artefatos de planejamento elaborados são validados. Se eles forem válidos, o processo segue para o subprocesso ``Realizar Aceitação da Fase''. Se não, o processo segue para o subprocesso ``Verificar Aplicação de Sanções''. O primeiro realiza a aceitação da fase de planejamento, enquanto o último verifica se cabem sanções, aplica-as e encaminha o processo para que sejam realizados novas reuniões de planejamento de sorte que os artefatos sejam reavaliados, corrigidos e completados.

Após o aceite da fase de planejamento, o processo segue para a Fase de Desenvolvimento. É nesta fase que os \emph{sprints} de desenvolvimento serão realizados. Dessa forma, os artefatos de saída do \nameref{sec:fluxo-sprint} também passarão pelo \nameref{sec:fluxo-rat} e pelo processo de validação tal qual ocorreu com os artefatos de planejamento da fase anterior. A diferença é que o subprocesso ``Realizar Aceitação da Fase'' pode aceitar a \emph{sprint} como válida, mas mesmo assim exigir que uma nova \emph{sprint} seja realizada. E haverão tantas \emph{sprints} quanto forem necessárias até que se decida que não são necessárias mais \emph{sprints} de desenvolvimento.

Em seguida, o processo chega na Fase de Encerramento. Nesta fase realiza-se uma ``Reunião de Encerramento'' que deve gerar ``Artefatos de Encerramento''. Novamente ocorre um ciclo de ateste técnico e validação até a aceitação da fase ser realizada e o processo chegar ao seu fim. 

\subsection{Subprocesso Sprint}
\label{sec:fluxo-sprint}

    \begin{figure}[htb]
    \begin{center}
        \includegraphics[scale=0.5]{fig/BPMN-SPRINT.png}
    \end{center}
    \caption{Subprocesso Sprint}
    \label{fig-sprint}
    \end{figure}

A figura \ref{fig-sprint} apresenta o subprocesso \emph{Sprint} que é onde a Fábrica de Software realiza o desenvolvimento do sistema propriamente dito. 

O processo começa com a ``Reunião de Planejamento da Sprint''. Esta é uma reunião onde a \emph{Sprint Backlog} para esta \emph{sprint} será definida.

Em seguida ocorrem duas atividades em paralelo: Enquanto a fábrica de software desenvolve e entrega os entregáveis definidos na reunião, os líderes da CLDF (TI e Negócios) acompanham esse desenvolvimento.

Finalmente, o processo termina após a realização da ``Reunião de Encerramento da Sprint''.


\subsection{Subprocesso Realizar Ateste Técnico}
\label{sec:fluxo-rat}

    \begin{figure}[ht]
    \begin{center}
        \includegraphics[scale=0.5]{fig/BPMN-RAT.png}
    \end{center}
    \caption{Subprocesso Realizar Ateste Técnico}
    \label{fig-rat}
    \end{figure}

Já na figura \ref{fig-rat} apresentamos o subprocesso que ilustra como os atestes técnicos das diversas fases são realizados. 
Basicamente são dois tipos de ateste: Serviço de Medição e Serviço de Qualidade. Sua realização não é obrigatória e a decisão de realizar esses serviços, ou não, é tomada pelos líderes de TI e Negócios.

\subsection{Subprocesso Realizar Reuniões de Planejamento}
\label{sec:fluxo-rrp}

    \begin{figure}[ht]
    \begin{center}
        \includegraphics[scale=0.5]{fig/BPMN-RRP.png}
    \end{center}
    \caption{Subprocesso Realizar Reuniões de Planejamento}
    \label{fig-rrp}
    \end{figure}

O Subprocesso Realizar Reuniões de Planejamento é mostrado na figura \ref{fig-rrp}. Inicialmente, uma ou mais reuniões entre as Áreas de TI e de Negócios da \CLDF são necessárias para que estas áreas definam seus respectivos líderes. Nessas reuniões as Histórias de Usuário do ``Product Backlog'' já podem começar a ser criadas. 

Em seguida, deve-se preparar uma ou mais reuniões entre os líderes da CLDF e os representantes das Fábricas de Software e de Qualidade. Essas reuniões deverão gerar os ``Artefatos de Planejamento''. Esses artefatos são acordos, como por exemplo, critérios de aceitação de qualidade, etc. 

Quando todos os artefatos forem definidos, o subprocesso se encerra. 



