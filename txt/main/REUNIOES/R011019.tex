\section{Resultados da Reunião do dia 01/10/19}

Em ordem cronológica, listo os itens discutidos na reunião e defino as conclusões: 

\subsection{Cabeçalho do documento}

    De acordo com João Ferreira, só pode ter 3 linhas:
    
    \begin{env-cinza}
        Câmara Legislativa do Distrito Federal
    
        Coordenadoria de Modernização e Informática - CMI
    
        Seção de Administração de Sistemas - SEASI
    \end{env-cinza}
    
    \TODO{Conferir}
    
\subsection{Processo  ``Entregar e Implantar'' do Subprocesso Sprint}

    \begin{env-aceitar}
        Mudar o nome para ``Entregar'' de forma a não confundir com homologação.
    \end{env-aceitar}

    \begin{envtodo}{Homologação e Produção da Release}
        Não existe no processo o momento de colocar a Release para Homologação e nem o momento de colocar um Release Homologado para produção. Preciso me preocupar com isso.
        \begin{enumerate}
            \item Precisamos diferenciar no processo a ``sprint'' da ``release''. Após algumas sprints temos uma release que é um entregável que já funciona. 
           
            \item A release deve ser colocada em ambiente de homologação pela Fábrica de Software. Assim, podemos testar e etc internamente.
           
            \item Uma release que passou pelo ``\nameref{sssec-sub-rat}'' e foi validado deve entrar em produção em algum momento, mas quem faz isso é a Seção de Infraestrutura da CLDF (ou não
            
            \item \textbf{O momento para isso é ao Aceitar a Fase (descrever isso lá)}
            
        \end{enumerate}
    \end{envtodo}

    \TODO{A reunião de retrospectiva acontece na Reunião de Encerramento da Sprint? Acho que sim!}
    
    \TODO{Acho que o macroprocesso já foi aprovado.}
    
    \WR{PAREI EM 10m47s}
    
\subsection{Subprocesso Sprint}
   
   A idéia do Jefferson é só envolver o Líder de Negócios no início e no final da Sprint nas reuniões. Ou seja, posso fazer isso ficar claro no processo criando uma raia dó para o Líder Técnico mostrando que só ele Acompanha a Sprint.
   
\subsection{Introdução}   

    Dizer em algum lugar que nosso processo é uma adaptação do SCRUM. Não é o SCRUM. É uma adaptação dele. Pois temos a figura do Líder Técnico e o do Líder de Negócios. 
    
\subsection{Realizar Ateste Técnico}       

    O PO não deve falar com a Fábrica sozinho. O Líder Técnico seria o único responsável por essa fase de Aceitação. 
    
    \TODO{Alterar no diagrama e no texto escrever isso em algum lugar.}
    
    Vamos deixar de onerar o tempo do Líder de Negócios. O Líder de Negócios só é requisitado em alguns momentos. A gente chama ele só quando for preciso. O resto é com o Líder Técnico.
    
\subsection{Artefatos de Planejamento}    

    Dizer em algum momento que os artefatos que elenquei são Artefatos Sugeridos. Dizer também em algum lugar que os artefatos necessários em cada projeto são definidos pelo Líder Técnico nas reuniões de planejamento.
    
    \TODO{Conferir se no TR existe o dizer que diz que o MDS será apresentado na primeira reunião e etc...}
    
    \begin{envtodo}{Cláusula TR}
        Conferir também se no TR existe a Cláusula que diz que o MDS pode ser alterado desde que isso seja avisado com 60 dias de antecedência.
    \end{envtodo}
    
    