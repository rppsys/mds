
\subsection{Documentação de sistemas legados}

São demandas para elaboração de documentação e (ou) atualização de documentação de sistemas legados. A documentação mínima necessária será: manual do sistema, manual do usuário, Modelo de Entidade e Relacionamento (MER), código‐fonte documentado e ajuda do sistema (preferencialmente em modo online).



    
    \begin{env-cinza}
    
    \textbf{Exemplo}
    
    Após a definição dos líderes mas ainda, sem envolver os representantes das fábricas, o Líder Técnico se reúne com o Líder de Negócios e demais interessados no projeto para desenvolver uma primeira versão do Documento de Visão. 
    
    Em seguida, outras reuniões são realizadas usando técnicas de \emph{brainstorm} para começar a se criar os primeiros itens do \PB.
    
    Com o ``Documento de Visão'' definido e um \PB inicial criado, os Líderes convocam os representantes das fábricas para uma primeira reunião. Nesta reunião, os Líderes apresentam o ``Documento de Visão'' para os representantes das fábricas e apresentam os primeiros itens do \PB. Neste momento, as fábricas podem contribuir elicitando e sugerindo mais itens para o \PB. 
    
    Em outra reunião, a pedido do Líder Técnico, o representante da Fábrica de Qualidade pode sugerir artefatos que estabeleçam critérios de qualidade que a Fábrica de Software deverá seguir. Os envolvidos conversam a respeito.
    
    Em outra reunião, os Líderes apresentam para as fábricas outros artefatos que julgarem necessários. O apêndice X apresenta uma tabela de artefatos sugeridos e o momento em que podem ser usados.  
    
    Todas as dúvidas das partes são esclarecidas e uma ou mais ``Atas de Reunião de Planejamento'' são assinadas por todos os participantes. Essas ``Atas'' estabelecem tudo o que foi acordado nessas reuniões indicando as responsabilidades de cada uma das partes. Um exemplo de ``Ata de Reunião'' é apresentada no apêndice X.
    \end{env-cinza}
    
    
    % -----------------------------------------------------
    \subsubsection{Documento de Validação}
    \label{sec:art-docvalida}
    % -----------------------------------------------------    
    Este é um documento que pode ser gerado pelos Líderes (Técnico e Negócios) apresentando o resultado do subprocesso ``Realizar Ateste Técnico'' de cada fase do processo (planejamento, desenvolvimento e encerramento).
    
    Ele informa se os produtos atestados foram validados ou não, e se não, porque?
    
    % -----------------------------------------------------    
    \subsubsection{Convocação de Reuniões de Planejamento}
    \label{sec:art-convocreuniaoplan}
    % -----------------------------------------------------    
    
    A convocação de uma reunião geralmente é formalizada por meio de um e-mail destinado aos interessados. Mas pode ocorrer de outras formas como, por exemplo, um memorando.
    
    
    
                    % Cabeçalho
            \rowcolor{lightgray!10}
            \multirow{2}{*}{ARTEFATO} &
            \multicolumn{5}{c|}{\textbf{ELABORAÇÃO}} \\ \cline{2-6}
            & LN & LT & FS & FQ & FM \\ \hline
