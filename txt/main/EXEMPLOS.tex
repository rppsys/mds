\section{Testes Latex}

\setlength{\columnsep}{0.2in}
\begin{multicols}{2}
    \begin{itemize}
        \item	Direciona o projeto, prioriza o que será feito, conversa com as áreas envolvidas;
        \item	Garante o patrocínio do projeto;
        \item	Normalmente é um servidor da área de negócio (gestora do sistema);
        \item	Aceita, rejeita ou pede ajustes nas entregas, do ponto de vista do negócio;
        \item	Assina um termo de recebimento das funcionalidades.  
        \item	Garante que as demandas passadas à empresa atendem aos usuários e ao direcionamento da Alta Administração.
    \end{itemize}    
\end{multicols}


% Exemplos de Referências Cruzadas
\begin{itemize}    
    \item \hyperref[sec-papeis]{ \underline{Esse é um link para secao papeis}} e note que o comando aqui usa colchetes como primeiro argumento
    
    \item \autoref{sec-papeis} cria a subseção automaticamente
    \item \nameref{sec-papeis} é uma outra forma
    \item Olhar este         \href{https://tex.stackexchange.com/questions/180571/making-clickable-links-to-sections-with-hyperref}{ \underline{link do stackoverflow}} e essa 
    documentação do overleaf em \url{https://www.overleaf.com/learn/latex/Hyperlinks}
\end{itemize}
        


Tabela Colorida com 3 Colunas

    {\rowcolors{3}{green!80!yellow!50}{green!70!yellow!40}    
    \begin{tabular}{ |p{3cm}|p{3cm}|p{3cm}|  }
        \hline
        \multicolumn{3}{|c|}{Participantes da Reunião de Planejamento} \\
        \hline
        Country Name or Area Name & ISO ALPHA 2 Code &ISO ALPHA 3 \\
        \hline
        Afghanistan & AF &AFG \\
        Aland Islands & AX   & ALA \\
        Albania &AL & ALB \\
        Algeria    &DZ & DZA \\
        American Samoa & AS & ASM \\
        Andorra & AD & AND   \\
        Angola & AO & AGO \\
        \hline
    \end{tabular}
    }
    
    {\rowcolors{3}{green!80!yellow!50}{green!70!yellow!40}    
    \begin{tabular}{ |p{2cm}|p{2cm}|p{2cm}| p{2cm} | }
        \hline
        \multicolumn{4}{|c|}{Participantes da Reunião de Planejamento} \\
        \hline
        Instituição &  Área & Ator & Papel \Scrum  \\
        \hline
        CLDF & Negócios & Líder de Negócios & \PO \\
        CLDF & TI & Líder Técnico & - \\
        Fábrica de Software & - & \SM da Fábrica de Software & \SM \\
        
        \hline
    \end{tabular}
    }
    
    
    
    \setlength{\arrayrulewidth}{1pt}
    \setlength{\tabcolsep}{12pt}
    \renewcommand{\arraystretch}{1}    
    
        \begin{table}
            \begin{center}
                \begin{tabular}{ |p{2cm}|p{6cm}|p{4cm}| }
                    \hline
                    \rowcolor{lightgray} \multicolumn{3}{|c|}{Artefatos} \\
                    \hline
                    \rowcolor{lightgray!20} Etapa & Artefato &  Responsáveis  \\
                    \hline

% PLANEJAMENTO
\multirow{8}{*}{Planejamento} 
 & Documento de Visão & CLDF (TI e Negócios) \\ \cline{2-3}
 & Documento de Definição de Líder Técnico e Líder de Negócios & CLDF (TI e Negócios) \\ \cline{2-3}
 & Glossário de termos técnicos do negócio & Área de Negócios \\ \cline{2-3}
 & Product Backlog & FS, FQ, LT, LN \\ \cline{2-3}
 & Proposta de Identidade Visual & Fábrica de Software \\ \cline{2-3}
 & Relatório de Verificação de Requisitos & Fábrica de Qualidade \\ \cline{2-3}
 & Estratégia de Teste & Fábrica de Qualidade \\ \cline{2-3}
 & Ata de Reunião de Planejamento & FS, FQ, CLDF \\ 
 
 \hline
 
% DESENVOLVIMENTO

\multirow{18}{*}{Desenvolvimento}

& Documento de Visão & CLDF (TI e Negócios) \\ \cline{2-3}
& Documento de Definição de Líder Técnico e Líder de Negócios & CLDF (TI e Negócios) \\ \cline{2-3}




                    \hline
                \end{tabular}
                \caption{\label{tab:art-plan} Artefatos sugeridos}
            \end{center}
        \end{table}
        
        
\DETALHAR
{Tipo}
{Nome}
{
Descrever os objetivos de forma resumida.
}
{
Participante 1

Participante 2
}    
{
Entrada 01

Entrada 02

Entrada 03
}
{
Saida 01

Saída 02
}    
{
    \textbf{Outras informações:}
    \setlength{\columnsep}{0.2in}
    \begin{multicols}{3}
        \begin{enumerate}
            \item Coisa 1
            \item Coisa 2
            \item Coisa 3
        \end{enumerate}
    \end{multicols}
}    
{fig/artefato-docvisao2.jpg}
{lightgray!50}

%########################################
\DETALHAR
{Tipo}
{Nome}
{
Descrever os objetivos de forma resumida.
}
{
Participante 1

Participante 2
}    
{
Entrada 01

Entrada 02

Entrada 03
}
{
Saida 01

Saída 02
}    
{
    \textbf{Outras informações:}
    \setlength{\columnsep}{0.2in}
    \begin{multicols}{3}
        \begin{enumerate}
            \item Coisa 1
            \item Coisa 2
            \item Coisa 3
        \end{enumerate}
    \end{multicols}
}    
{fig/artefato-docvisao2.jpg}
{lightgray!50}
%########################################
\DETALHAR
{Tipo}
{Nome}
{
Descrever os objetivos de forma resumida.
}
{
Participante 1

Participante 2
}    
{
Entrada 01

Entrada 02

Entrada 03
}
{
Saida 01

Saída 02
}    
{
    \textbf{Outras informações:}
    \setlength{\columnsep}{0.2in}
    \begin{multicols}{3}
        \begin{enumerate}
            \item Coisa 1
            \item Coisa 2
            \item Coisa 3
        \end{enumerate}
    \end{multicols}
}    
{fig/artefato-docvisao2.jpg}
{lightgray!50}
%########################################
\DETALHAR
{Tipo}
{Nome}
{
Descrever os objetivos de forma resumida.
}
{
Participante 1

Participante 2
}    
{
Entrada 01

Entrada 02

Entrada 03
}
{
Saida 01

Saída 02
}    
{
    \textbf{Outras informações:}
    \setlength{\columnsep}{0.2in}
    \begin{multicols}{3}
        \begin{enumerate}
            \item Coisa 1
            \item Coisa 2
            \item Coisa 3
        \end{enumerate}
    \end{multicols}
}    
{fig/artefato-docvisao2.jpg}
{lightgray!50}
%########################################
        